\section{Conclusiones y recomendaciones}
A partir de este trabajo se obtienen las siguientes conclusiones.
\begin{itemize}
\item El uso de los ejemplos brindados por la biblioteca \texttt{libopencm3} sirvió de ayuda para realizar las funciones del sismografo, ya que se logró ver elementos en la pantalla LCD, y a partir de esto se usaron los bloques de código necesarios para mostrar un simple texto, añadirle color, posición y otros detalles, ya con esto fue un gran avance y poder implementar el giroscopio con base a las demás configuraciones de \texttt{gpio} y sensibilidad en los ejes para mostrar los valores en cada eje.
\item Se aprendió como implementar una comunicación entre una nube y un microcontrolador.
\item A pesar de que no se lograron todos los puntos estipulados por el enunciado, se considera de que se lograron los más importantes. 
\item A partir de la tensión de salida en la batería (\SI{3.21}{\volt}) y haber realizado ajustes en funciones como \texttt{read\_adc\_naiive} se logró mostrar esta tensión en la pantalla y encender o apagar el LED respetando el umbral previamente dado.
\end{itemize}

Las recomendaciones de este trabajo son las siguientes:
\begin{itemize}
\item Verificar de que los datos se estén enviando de acuerdo al protocolo, varias veces se tuvieron problemas de que los datos no eran compatibles y esto era porque se le estaban enviando strings o ints y se esperaban bytes. El cálculo de un tiempo adecuado para que se dé la comunicación es muy importante y fue uno de los problemas más grandes que se tuvo con este laboratorio. La lectura de los ejemplos proporcionado fue de suma importancia para poder completar el laboratorio.
\item Probar los ejemplos que vienen en la librería \texttt{libopencm3}, esto ayuda a entender la funcionalidad de los bloques de código.
\item Realizar muchas pruebas y error con los ejemplos.
\item Crear el proyecto en la misma carpeta donde están los ejemplos, esto para no tener ningún problema a la hora de usar el makefile.
\item Tener mucho cuidado en las conexiones para cuidar el equipo de trabajo.
\end{itemize}

