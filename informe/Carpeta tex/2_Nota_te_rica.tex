\section{Nota teórica}
En esta sección se describen los componentes principales que se utilizaron para el desarrollo de un sismógrafo.
\subsection*{STM32F429 Discovery kit}
Este microcontrolador permite a los usuarios desarrollar fácilmente aplicaciones de alto desempeño. Incluye un ST-LINK/V2 embebido como una herramienta de depuración, una SRAM externa de 64-Mbit, un ST MEMS giroscopio, un USB OTG conector AB, LEDs y botones. Algunas de las características generales se resumen a continuación.

\subsubsection*{Características generales}
Las características más importantes de este mcu se mencionan a continuación:
\begin{multicols}{2}
 \begin{itemize}
    \item 2.4 QVGA TFT LCD.
    \item 64-Mbit SDRAM.
    \item USB OTG con conector Micro-AB.
    \item Header para LQFP144 I/Os.
    \item Sensor de movimiento I3G4250D, Giroscopio ST MEMS de 3-ejes-
    \item On-board ST-LINK/V2-B.
    \item Alimentación por USB o fuente externa de \SI{3}{\volt} o \SI{5}{\volt}.
    \item 2  push-button (Usuario y reset).
    \item Core: ARM 32 bits Cortex-M4 con FPU (RISC).
    \item Debug: SWD, JTAG.
    \item Trabaja en frecuencia de \SI{180}{\mega\Hz}
    \item 168 I/O con capacidad de interrupción.
    \item 2MB flash, 256 KB SRAM.
    \item Controlador LCD-TFT.
    \item 21 interfaces de comunicaciones(I2C,USART,SPI,SAI,CAN).
    \item Low Power.
    \item Conectividad avanzada USB 2.0.
    \item Intefaz de camara.
    \item 2x12bit convertidor D/A.
    \item True RNG.
    \item CRC.
    \item 6 LEDS: LD1 (USB Comms), LD2(3.3V PowerOn, 2 LEDS de ususario (LD3 y LD4), 2 LEDS USB OTG (LD5 y LD6).
    \item Controladores DMA.
    \item 17 timers: 12 timers de 16bit, 2 de 32bit de hasta 180MHz, c/u con 4IC/OC/PWM.
\end{itemize}   
\end{multicols}

\subsubsection*{Diagrama de bloques}
En la figura \ref{fig1} se muestra el diagrama de bloques del STM32F429.
\begin{figure}[H]
\centering
\includegraphics[width=.55\linewidth]{Imagenes/1.png}
 \caption{Diagrama de bloques del STM32F429 . Tomado de \cite{web}.}
 \label{fig1}
\end{figure}
\subsubsection*{Diagrama de pines}
Luego, el diagrama de pines de este mcu se presenta en la figura \ref{fig2}
\begin{figure}[H]
\centering
\includegraphics[width=.55\linewidth]{Imagenes/2.png}
 \caption{Diagrama de pines del STM32F429. Tomado de \cite{web}.}
 \label{fig2}
\end{figure}
\subsubsection*{Características eléctricas}
Las siguientes tablas resumen las características eléctricas de este microcontrolador.
\begin{figure}[H]
\centering
\includegraphics[width=.55\linewidth]{Imagenes/3.png}
 \caption{Detalles del voltaje del mcu. Tomado de \cite{web}.}
 \label{fig3}
\end{figure}

\begin{figure}[H]
\centering
\includegraphics[width=.55\linewidth]{Imagenes/4.png}
 \caption{Detalles de la corriente en el mcu. Tomado de \cite{web}.}
 \label{fig4}
\end{figure}


\subsection*{Periféricos utilizados}
\subsection*{Componentes electrónicos complementarios}
% quiza mencionar la ayuda de la protoboard, en realidad fue como lo único.
Es un circuito que lo compone una electrónica básica (así lo resume la tabla \ref{table_2}), solo se usó una protoboard y 3 resistencias en total, una batería de \SI{9}{\volt}, esto para realizar un divisor de tensión con el objetivo de alimentar a la placa STM32249 Discovery Kit con \SI{5}{\volt}. Así, se sabe que $v_{out} \approx \SI{5}{\volt}$
\begin{itemize}
\item $R_1 = \SI{20}{\kilo\ohm}$
\item $R_2 = \SI{18}{\kilo\ohm}$
\item $v_{in} =  \SI{9}{\volt}$
\end{itemize}
Aplicando el divisor de tensión se tiene que:
\begin{equation}
v_{out} = \SI{9}{\volt} \cdot \frac{  \SI{20}{\kilo\ohm} }{ \SI{20}{\kilo\ohm}+\SI{18}{\kilo\ohm}} \approx \SI{4.737}{\volt}
\label{eq1}
\end{equation}
De la ecuación \ref{eq1}, se demuestra que con estas magnitudes es posible alimentar la placa sin sobrepasar el umbral.
\subsection*{Lista de componentes}
La lista de componentes fueron consultados en \cite{web2} disponibles
\begin{table}[H]
\caption{Lista de equipos}
\label{table_2}
\begin{center}
\begin{tabular}{r|cc}
\hline
\textbf{Componente}&\textbf{Cantidad}&\textbf{Precio}\\
 \hline
STM32F429 Discovery Kit& 1 & 83\$ \\ \hline 
Resistencias \SI{20}{\kilo\ohm}&2 & 0.4\$ \\ \hline 
Resistencias \SI{18}{\kilo\ohm}&1 & 0.2\$ \\ \hline 
Protoboard &1 &10\$ \\ \hline 
Broche porta pila &1 &0.5\$ \\ \hline 
Baterías \SI{9}{\volt} & 2& 2\$ \\ \hline 

 \textbf{Total}& & 96.1\$ \\
 \hline
\end{tabular}
\end{center}
\end{table}

\subsection*{Diseño del circuito}
El diagrama mostrado en la figura \ref{DF_S}, resume el funcionamiento del sismógrafo.
\input{sch1.tex}
\newpage
